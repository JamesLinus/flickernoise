\documentclass[11pt,a4paper]{article}
\usepackage{graphicx}
\usepackage{colortbl}
\usepackage{fullpage}
\usepackage{parskip}
\usepackage{booktabs}
\usepackage{epsfig}
\usepackage{moreverb}
\usepackage{nicefrac}

\newenvironment{expose}{\vskip3mm\qquad\begin{raggedright}}{%
\end{raggedright}\vskip3mm}

\begin{document}

\title{Using MIDI in Flickernoise Patches}
\author{Werner Almesberger \\
  {\normalsize werner@almesberger.net}}

\maketitle


% -----------------------------------------------------------------------------


\section{Introduction}

This document describes the architecture of the MIDI control @@@

Note that this work is still in progress. For example,
\begin{itemize}
  \item the MIDI device specifications should not have to be part of
    patches,
  \item Flickernoise should only consider devices that are really present, and
  \item we should support other events that just MIDI control message,
    such as other MIDI message types, DMX, keyboard, IR remote, etc.
\end{itemize}


% -----------------------------------------------------------------------------


\section{Quick start}

While the finer details of MIDI controls can get complicated, the
elements in the following example are often all that is needed to
make full use of many MIDI devices:

\begin{listing}{1}
midi "Gizmo" {
	main = fader(102);
	aux = pot(103);
	but1 = button(16);
	but2 = button(17);
}

sensitivity = linear(main);

per_frame:
	wave_scale = sensitivity*10;
	sensitivity = sensitivity*0.99;
\end{listing}

In lines 1 through 6 we describe the controls the MIDI device called
``Gizmo'' provides. In this case, we have one fader, one potentiometer,
and two buttons. We assign them names that represent their role: {\tt main}
for the principal fader, {\tt aux} for the potentiometer, and so on.

The numbers are the respective MIDI controller numbers the device uses.
They can be found with the MIDI monitor function of the ``MIDI settings''
dialog in Flickernoise.

In line 8, we bind the {\tt main} control to a variable. This variable is
then used in per-frame and per-vertex equations. As line 12 shows, one
can also change this variable, e.g., to make the sensitivity slowly decay
if there is no input from the MIDI device.


% -----------------------------------------------------------------------------


\section{Architecture}


% - - - - - - - - - - - - - - - - - - - - - - - - - - - - - - - - - - - - - - -


\subsection{MIDI control message}

- MIDI: 0-127


% - - - - - - - - - - - - - - - - - - - - - - - - - - - - - - - - - - - - - - -


\subsection{...}


% -----------------------------------------------------------------------------

\section{Configuration process}

% - - - - - - - - - - - - - - - - - - - - - - - - - - - - - - - - - - - - - - -


\subsection{Device database}

Device specifications are added to the device database with 
- global section

\begin{listing}{1}
midi "dev1" {
	main = fader(1, 102);
	aux = pot(1, 103);
	but1 = button(1, 16);
	but2 = button(1, 17);
	tog = toggle(1, 24);
}
\end{listing}


% - - - - - - - - - - - - - - - - - - - - - - - - - - - - - - - - - - - - - - -


\subsection{Binding}

In order to use a control in a patch, we have to establish a connection
between the control element (in the device database) and a patch
variable. We call such a variable a {\em control variable}.

Control variables are bound with a construct that looks like a
variable assignment:

\begin{expose}
{\tt {\em control\_variable} = {\em function}({\em element\_name});}
\end{expose}

{\em control\_variable} can be a pre-defined per-frame or per-vertex
variable or it can be a user-defined variable. Variables that are
updated by Flickernose itself, e.g., {\tt time} or {\tt bass}, cannot
be used as control variables.

{\em function} describes how the patch expects the control to behave.
Flickernoise then tries to adapt the behaviour of the actual device
to what the patch expects. The following functions are available:

\begin{description}
  \item[\tt range]
    The control variable has a value between 0 and 1, depending on the
    setting of the device. This is commonly used for faders and
    potentiometers.
  \item[{\tt unbounded}, {\tt cyclic}]
    These are special functions used with some rotary encoders. They
    are described in detail in section \ref{diff}. With other control
    elements, they behave just like {\tt range}.
  \item[\tt button]
    The control variable receives the value 1 when the button is
    pressed and returns to zero when it is released. When applied to
    elements that send values between 0 and 1, the value is rounded
    and the control acts like a {\tt toggle}.
  \item[\tt toggle]
    The value of the control variable changes between 0 and 1 each
    time the element -- typically a button -- is actuated. 
\end{description}

{\em element\_name} is the name of the control element, as in the
device database.

\begin{figure}[!tb]
\begin{center}
\includegraphics{bind}
\end{center}
\caption{When binding a control variable, information from the device
  database, a translation map, and the table of patch variables is
  combined into a stimulus entry.}
\label{bind}
\end{figure}


% - - - - - - - - - - - - - - - - - - - - - - - - - - - - - - - - - - - - - - -


\subsection{Event processing}

\begin{figure}[!tb]
\begin{center}
\includegraphics{stim}
\end{center}
\caption{When a MIDI control message is received, Flickernoise looks
  for a matching stimulus and processes the value accordingly.}
\label{stim}
\end{figure}

Figure \ref{stim} illustrates how events are translated into changes
of control variables.

Each time a MIDI control message arrives, Flickernoise looks for a
stimulus that matches the event type, i.e., MIDI control, the channel
number and the controller number (1). If no matching entry exists,
the event is ignored.

The stimulus tells Flickernoise which translator to use and where
to store the result. In our example, we have a {\tt pot} to {\tt range}
translation, which is simply a division by 127 (2). The result,
$\nicefrac{28}{127}=0.220$, is stored in the location of the variable
{\tt var} (3).


% - - - - - - - - - - - - - - - - - - - - - - - - - - - - - - - - - - - - - - -


\subsection{Writing to control variables}

- pfv
- pvv
- continuity

exercised to avoid any of the per-defined variables that are updated

% -----------------------------------------------------------------------------


\section{Control devices}

In the section below, we describe the various control elements, their
declaration, and their behaviour. In the example, we show the physical
state of the element, the value a MIDI control may typically send for
the element in that state, and then the resulting values for the
various translations.

Since {\tt range}, {\tt unbounded}, and {\tt cycle} only differ from
each other in one case, they are usually abbreviated as just
``{\tt range}, $\ldots$''.


% - - - - - - - - - - - - - - - - - - - - - - - - - - - - - - - - - - - - - - -


\subsection{Faders}

Faders are slide potentiometers that normally cover the whole
range of values a MIDI controller can send: 0--127. They
retain their position when released.

Fader elements of a device are declared with {\tt fader()} and
are typically bound with {\tt range()}:

\begin{listing}{1}
midi ... {
	name = fader(...);
}

var = range(name);
\end{listing}

The following example shows how Flickernoise maps faders to control
variables. We start with the fader in the 0\% position, move it to
the 40\%, then 60\%, and the 100\% position. At the end, we return
it to the 0\% position.

\begin{expose}
\begin{tabular}{lcccccl}
  \raisebox{6mm}{User input} &
  \includegraphics{fader-0.pdf} &
  \includegraphics{fader-40.pdf} &
  \includegraphics{fader-60.pdf} &
  \includegraphics{fader-100.pdf} &
  \includegraphics{fader-0.pdf} \\
  \cmidrule(r){1-6}
  MIDI value &
  & 51 & 76 & 127 & 0 \\
  \midrule
  Translation
  & 0 & 0.4 & 0.6 & 1 & 0 & \tt range, $\ldots$ \\
  & 0 & 0   & 1   & 1 & 0 & \tt button, toggle \\
\end{tabular}
\end{expose}

The mapping is quite straightforward: {\tt range}, {\tt unbounded},
and {\tt cycle} produce a value from 0 to 1 corresponding to the
position of the knob. {\tt button} and {\tt toggle} produce 0 if
the knob is the lower half of the range, 1 if it is in the upper half.


% - - - - - - - - - - - - - - - - - - - - - - - - - - - - - - - - - - - - - - -


\subsection{Rotary potentiometers}

Rotary potentiometers work exactly like faders except that they are
declared with {\tt pot()}:\footnote{The current implementation does
not distinguish at all between {\tt fader} and {\tt pot}, but they
may be represented with different symbols in a future GUI.}

\begin{listing}{1}
midi ... {
	name = pot(...);
}

var = range(name);
\end{listing}

They have mechanical stops at the beginning and at the end of
their range, which distinguishes them from the rotary encoders described
in the next section.

The example below shows a potentiometer that travels over an angle
of $270^{\circ}$:

\begin{expose}
\begin{tabular}{lccccl}
  \raisebox{6mm}{User input} &
  \includegraphics{pot-0.pdf} &
  \includegraphics{pot-90.pdf} &
  \includegraphics{pot-180.pdf} &
  \includegraphics{pot-270.pdf} \\
  \cmidrule(r){1-5}
  MIDI value &
  & 42 & 85 & 127  \\
  \midrule
  Translation
  & 0 & 0.33 & 0.67 & 1 &  \tt range, $\ldots$ \\
  & 0 & 0    & 1   & 1 &  \tt button, toggle \\
\end{tabular}
\end{expose}


% - - - - - - - - - - - - - - - - - - - - - - - - - - - - - - - - - - - - - - -


\subsection{Rotary encoders acting as potentiometers}

Rotary encoders look similar to potentiometers but differ from them
by not having a mechanical stop. This means that they can be turned
indefinitely in the same direction.

MIDI devices usually emulate the behaviour of potentiometers by ignoring
any turns at the end of the value range. When the direction is reversed,
the values change immediately.

Rotary encoders acting as potentiometers are also declared with
{\tt pot} and bound with {\tt range}.

The example below shows a rotary encoder covering the full value range
in one $360^{\circ}$ turn that is first turned $450^{\circ}$ clockwise
and then $45^{\circ}$ counterclockwise:

\begin{expose}
\begin{tabular}{lcccccl}
  \raisebox{6mm}{User input} &
  \includegraphics{enc-0.pdf} &
  \includegraphics{enc-90.pdf} &
  \includegraphics{enc-270.pdf} &
  \includegraphics{enc-450.pdf} &
  \includegraphics{enc-405.pdf} \\
  \cmidrule(r){1-6}
  MIDI value &
  & 32 & 96 & 127 & 121  \\
  \midrule
  Translation
  & 0 & 0.25 & 0.75 & 1 & 0.875 & \tt range, $\ldots$ \\
  & 0 & 0    & 1    & 1 & 1     & \tt button, toggle \\
\end{tabular}
\end{expose}

Note that it stops at MIDI value 127 at the third turn and the
remaining $90^{\circ}$ are ignored.


% - - - - - - - - - - - - - - - - - - - - - - - - - - - - - - - - - - - - - - -


\subsection{Push buttons}

Push buttons are activated by pressing them and they return to the
inactive state when released. Buttons can only be either fully on or
fully off, without intermediate values.

Push buttons are declared with {\tt button()} and bound with
{\tt button()} or {\tt toggle()}:

\begin{listing}{1}
midi ... {
	name = button(...);
}

var = button(name);
\end{listing}

The alternative {\tt toggle()} turns the control on when the button
is pressed the first time and then off again when pressed the second
time.

Example:

\begin{expose}
\begin{tabular}{lcccccl}
  \raisebox{5mm}{User input} &
  \includegraphics{button-up.pdf} &
  \includegraphics{button-down.pdf} &
  \includegraphics{button-up.pdf} &
  \includegraphics{button-down.pdf} &
  \includegraphics{button-up.pdf} \\
  \cmidrule(r){1-6}
  MIDI value &
  & 127 & 0 & 127 & 0\\
  \midrule
  Translation
  & 0 & 1 & 0 & 1 & 0 & \tt range, $\ldots$, button \\
  & 0 & 1 & 1 & 0 & 0 & \tt toggle \\
\end{tabular}
\end{expose}


% - - - - - - - - - - - - - - - - - - - - - - - - - - - - - - - - - - - - - - -


\subsection{Toggles}

A toggle button or switch flips from on to off each time it is
actuated. Toggles are declared with {\tt toggle()} and typically
bound with {\tt toggle()} as well:

\begin{listing}{1}
midi ... {
	name = toggle(...);
}

var = toggle(name);
\end{listing}

In the example below, we indicate whether the control is activated
by a lamp next to the button:

\begin{expose}
\begin{tabular}{lcccccl}
  \raisebox{6mm}{User input} &
  \includegraphics{toggle-off.pdf} &
  \includegraphics{toggle-act.pdf} &
  \includegraphics{toggle-on.pdf} &
  \includegraphics{toggle-deact.pdf} &
  \includegraphics{toggle-off.pdf} \\
  \cmidrule(r){1-6}
  MIDI value &
  & 127 & 0 & 127 & 0\\
  \midrule
  Translation
  & 0 & 1 & 1 & 0 & 0 & all \\
\end{tabular}
\end{expose}


% - - - - - - - - - - - - - - - - - - - - - - - - - - - - - - - - - - - - - - -


\subsection{One-way buttons}

Some devices have buttons that send a MIDI event only when pressed but
not when released. E.g.,

\begin{expose}
\begin{tabular}{lcccccl}
  \raisebox{5mm}{User input} &
  \includegraphics{button-up.pdf} &
  \includegraphics{button-down.pdf} &
  \includegraphics{button-up.pdf} &
  \includegraphics{button-down.pdf} &
  \includegraphics{button-up.pdf} \\
  \cmidrule(r){1-6}
  MIDI value &
  & 127 &  & 127 & \\
\end{tabular}
\end{expose}

Since the control variables in Flickernoise reflect the state of a
control and not events, we cannot directly use such devices. However,
by exploiting the ability to overwrite a control variable, we can
simulate a release as follows:

\begin{listing}{1}
midi ... {
	name = button(...);
}

var = button(name);

per_frame:
	/* use "var" */
	var = 0;
\end{listing}

When the button is pressed, it will set {\tt var} to 1. At the next
frame, this value can be used. At the end of the frame (and before
running per-vertex equations), it is reset to zero. {\tt var} thus
acts like a {\tt button} control variable where the button is always
pressed for exactly one frame duration.


% - - - - - - - - - - - - - - - - - - - - - - - - - - - - - - - - - - - - - - -


\subsection{Differential encoders}
\label{diff}

\begin{expose}
\begin{tabular}{lccccccl}
  \raisebox{6mm}{User input} &
  \includegraphics{dial-0.pdf} &
  \includegraphics{dial-90.pdf} &
  \includegraphics{dial-180.pdf} &
  \includegraphics{dial-360.pdf} &
  \includegraphics{dial-405.pdf} &
  \includegraphics{dial-315.pdf} \\
  \cmidrule(r){1-7}
  MIDI value &
  & 32 & 32 & 64 & 16 & 96 \\
  \midrule
  Translation
  & 0 & 0.25 & 0.5 & 1 & 1 & 0.75 & \tt range \\
  & 0 & 0.25 & 0.5 & 1 & 1.125 & 0.875 & \tt unbounded \\
  & 0 & 0.25 & 0.5 & 0 & 0.125 & 0.875 & \tt cyclic \\
%  & 0 & 1 & 1 & 0 & \tt button, toggle \\
\end{tabular}
\end{expose}

%   The control variable can assume any value, including negative
%   values and values greater than one. This can be used with
%   certain rotary encoders. If the device in question has no
%   meaningful way to input unbounded values, {\tt unbounded} acts
%   like {\tt range}.
%   The control variable has a value between 0 and 1, like with
%   {\tt range}, but if the control element tries to move outside the
%   range, the value wraps around. 
\end{document}
